% erklaerung.tex
\pagestyle{empty}

\normalsize


\begin{center}
{\bf Eidesstattliche Versicherung }
\end{center}
~\\
\noindent \rule{10cm}{0.4pt}   \hspace{1.5cm} \noindent \rule{3.5cm}{0.4pt}  \\
      {Name, Vorname}    \hspace{8.7cm} {Matrikelnummer}  

\vspace{0.5cm}
Ich versichere hiermit an Eides statt, dass ich die vorliegende Bachelorarbeit/Masterarbeit* mit
dem Titel

\vspace{0.5cm}
\noindent \rule{15cm}{0.4pt}

\vspace{0.5cm}

\noindent \rule{15cm}{0.4pt}

\vspace{0.5cm}
\noindent \rule{15cm}{0.4pt}
~\\
selbstst\"andig und ohne unzul\"assige fremde Hilfe erbracht habe. Ich habe
keine anderen als die angegebenen Quellen und Hilfsmittel benutzt sowie
w\"ortliche und sinngem\"aße Zitate kenntlich gemacht. Die Arbeit hat in
gleicher oder \"ahnlicher Form noch keiner Pr\"ufungsbeh\"orde vorgelegen.
~\\

\noindent \rule{5cm}{0.4pt}   \hspace{3.5cm} \noindent \rule{6cm}{0.4pt}  \\
      {Ort, Datum}    \hspace{6.5cm} {Unterschrift}  \vspace{0.2cm}

      ~ \hspace{8.5cm} {* Nichtzutreffendes bitte streichen}

\vspace{0.5cm}
{\bf Belehrung:}

Wer vors\"atzlich gegen eine die T\"auschung \"uber Pr\"ufungsleistungen
betreffende Regelung einer Hochschulpr\"ufungsordnung verst\"oßt, handelt
ordnungswidrig. Die Ordnungswidrigkeit kann mit einer Geldbuße von bis zu
50.000,00 € geahndet werden. Zust\"andige Verwaltungsbeh\"orde für die
Verfolgung und Ahndung von Ordnungswidrigkeiten ist der Kanzler/die Kanzlerin
der Technischen Universit\"at Dortmund. Im Falle eines mehrfachen oder
sonstigen schwerwiegenden T\"uschungsversuches kann der Prüfling zudem
exmatrikuliert werden. (§ 63 Abs. 5 Hochschulgesetz - HG - )

Die Abgabe einer falschen Versicherung an Eides statt wird mit Freiheitsstrafe bis zu 3 Jahren
oder mit Geldstrafe bestraft.

Die Technische Universit\"at Dortmund wird gfls. elektronische
Vergleichswerkzeuge (wie z.B. die Software „turnitin“) zur \"Uberpr\"ufung von
Ordnungswidrigkeiten in Pr\"ufungsverfahren nutzen.

Die oben stehende Belehrung habe ich zur Kenntnis genommen:
~\\

\noindent \rule{5cm}{0.4pt}   \hspace{3.5cm} \noindent \rule{6cm}{0.4pt}  \\
      {Ort, Datum}    \hspace{6.5cm} {Unterschrift}  \vspace{0.2cm}
